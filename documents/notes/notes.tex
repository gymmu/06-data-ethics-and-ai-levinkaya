\documentclass{article}

\usepackage[ngerman]{babel}
\usepackage[utf8]{inputenc}
\usepackage[T1]{fontenc}
\usepackage{hyperref}
\usepackage{csquotes}

\usepackage[
    backend=biber,
    style=apa,
    sortlocale=de_DE,
    natbib=true,
    url=false,
    doi=false,
    sortcites=true,
    sorting=nyt,
    isbn=false,
    hyperref=true,
    backref=false,
    giveninits=false,
    eprint=false]{biblatex}
\addbibresource{../references/bibliography.bib}

\title{Notizen zum Projekt Data Ethics}
\author{Levin Kaya}
\date{\today}

\begin{document}
\maketitle

\abstract{
    Dieses Dokument ist eine Sammlung von Notizen zu dem Projekt. Die Struktur innerhalb des
    Projektes ist gleich ausgelegt wie in der Hauptarbeit, somit kann hier einfach geschrieben
    werden, und die Teile die man verwenden möchte, kann man direkt in die Hauptdatei ziehen.
}

\tableofcontents

\section{Einleitung}


\section{Fragestellung}
Wie kann man sicherstellen, dass die Antworten, die von einer Künstlichen Intelligenz generiert werden, auf vertrauenswürdigen und nachvollziehbaren Quellen basieren?


\section{Quellen}
https://www.scribbr.de/haufig-gestellte-fragen/woher-bezieht-chatgpt-seine-informationen/
https://www.fluter.de/faq-woher-nimmt-chatgpt-sein-wissen
https://de.wikihow.com/Die-Sicherheit-von-ChatGPT
https://www.europarl.europa.eu/topics/de/article/20200827STO85804/was-ist-kunstliche-intelligenz-und-wie-wird-sie-genutzt



\printbibliography

\end{document}

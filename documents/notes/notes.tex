\documentclass{article}

\usepackage[ngerman]{babel}
\usepackage[utf8]{inputenc}
\usepackage[T1]{fontenc}
\usepackage{hyperref}
\usepackage{csquotes}

\usepackage[
    backend=biber,
    style=apa,
    sortlocale=de_DE,
    natbib=true,
    url=false,
    doi=false,
    sortcites=true,
    sorting=nyt,
    isbn=false,
    hyperref=true,
    backref=false,
    giveninits=false,
    eprint=false]{biblatex}
\addbibresource{../references/bibliography.bib}

\title{Notizen zum Projekt Data Ethics}
\author{Levin Kaya}
\date{\today}

\begin{document}
\maketitle

\abstract{
    Dieses Dokument ist eine Sammlung von Notizen zu dem Projekt. Die Struktur innerhalb des
    Projektes ist gleich ausgelegt wie in der Hauptarbeit, somit kann hier einfach geschrieben
    werden, und die Teile die man verwenden möchte, kann man direkt in die Hauptdatei ziehen.
}

\tableofcontents

\section{Einleitung}
Zu Beginn dieser Arbeit werde ich auf die allgemeine Frage, woher die KI ihre Informationen bezieht eingehen und um welche spezifischen Datenquellen es sich dabei handelt. 
Das sogenannten Large Language Modell (LLM) wird von einer gigantischen Menge an Informationen aus dem Internet gefüttert, darunter milliarden von Büchern, Artikeln, Websites und Posts. Die Qualität der Informationen variiert dabei stark, von seriösen Quellen wie beispielsweise der Online-Lexika oder wissenschaftlicher Lieratur bis hin zu völlig falschen Angaben aus Foren oder Social Media. Nutzer:innen können mit einem ChatBot kommunizieren und eine Eingabe erstellen wie z.B. ''Wie verbessere ich meinen Schlaf?'' Nach dem Begriff durchsucht der Bot das Netzwerk und kombiniert dies mit der Eingabe.



\section{Fragestellung}
Wie kann man sicherstellen, dass die Antworten, die von einer Künstlichen Intelligenz generiert werden, auf vertrauenswürdigen und nachvollziehbaren Quellen basieren?


\section{Informationsquelle der KI}

.







github copilot
raub
unkorrekte Informationsquelle
keine weitere kontrolle
lehrpersonen könnens nicht benutzen
lwhrpersonen können falsche infos weitergeben (sek,primar)
riesen problematik
bei generierung des textes keine genauigkeit
mensch auch keine garantie aufb korrektheit
kritisch hinterfragen, auch was lehrpersonen sagen
zerstört die vielfalt
ki lernt abbildung der daten
subsection möglich
ki wird trainiert aufgrund der daten
anpassen der parameter wichtig
\printbibliography


\end{document}

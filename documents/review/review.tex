\documentclass{article}

\usepackage[ngerman]{babel}
\usepackage[utf8]{inputenc}
\usepackage[T1]{fontenc}
\usepackage{hyperref}
\usepackage{csquotes}

\usepackage[
    backend=biber,
    style=apa,
    sortlocale=de_DE,
    natbib=true,
    url=false,
    doi=false,
    sortcites=true,
    sorting=nyt,
    isbn=false,
    hyperref=true,
    backref=false,
    giveninits=false,
    eprint=false]{biblatex}
\addbibresource{../references/bibliography.bib}

\title{Review des Papers "Ethik im Umgang mit Daten" von Stefani Anastasova \dots}
\author{Levin Kaya}
\date{\today}

\begin{document}
\maketitle

\abstract{
Dies ist meine Review der Arbeit zum Thema Ethik im Umgang mit Daten von Stefani Anastasova.


}



\section{Positive Aspekte}
Liebe Stefani, wie ich finde hast du eine umfassende Einführung in die Thematik der KI verfasst und es war auch übersichtlich. Es erklärt gut die grundlegenden Konzepte wie beispielsweise überwachtes und unüberwachtes Lernen. Ich finde dass gut daher, da es zum Verständnis der Lesers beiträgt.
Auch die Trainingsschritte hast du klar strukturiert. Du hast beschrieben, wie Training, Validierung und Testen funktioniert. Daher sieht man, dass du diese Prozesse, die eine KI durchlaufen muss, bevor sie einsatzbereit ist verstanden hast.Ich finde deine Fragestellung sehr schlau gewählt, da die Frage wie KI-Systheme den Datenschutz respektieren können sehr aktuell und von grosser Bedeutung ist. Du hast ChatGBT als konkretes Beispiel angewendet, somit machst du das Thema greifbarer, denn mehr Leute können sich etwas darunter vorstellen. Somit kann der Leser oder die Lerserin, die theoretischen Konzepte, die du davor erläutert hast, auf ein reales zu übertragen. Zudem hast du meiner Meinung nach gute Massnahmen zur Sicherrstellung des Datenschutzes gegeben ( Einwilligung, Transparenz, Zweckbindung und Datensicherheit). Sie zeigen praktische Wege zur Lösung der beschriebenen Problemen.

einige rfechtschreibfehler besonders in der zusammenfassung

\section{Negative Aspekte}


Nun werde ich auf einige Punkte der Arbeit eingehen, die meiner Meinung nach Verbesserungspotenzial besitzen oder mehr ausgebaut werden sollten. Die Herausforderungen des Datenschutzes werden nur oberflächlich behandelt. Es fehlt eine vertiefte Analyse davon, was für mögliche Risiken oder negative Auswirkungen bei der Verwendung von ChatGBT mitsichtragen könnten. Ausserdem finde ich, dass einige Abschnitte, vorallem im Kapitel 2 etwas dazu tendieren unstrukturiert zu sein, und könnten etwas klarer gegliedert werden. Somit wäre es einfacher als Leser oder Leserin dem Inhalt deiner Arbeit zu folgen. Im ersten Kapitel hingegen finden ich, dass du dies sehr gut gemacht hast. Du hättest mehr an aktuellen Quellen und Zitaten in deine Arbeit miteinbringen können, da es da vielleicht einen kleinen Mangel gab. Damit ist gemeintauf aktuelle Studien, Berichte oder Literatur zu verweisen, um die Argumente, die du hervorgebracht hast zu untermauern. Das könnte ein Faktor  zur erhöhung der Glaubwürdigkeit sein, in deiner Arbeit. Zuletzt ist mir noch aufgefallen, dass es einen fehlenden Abschnitt gibt, der allenfalls noch weiter auf die ethischen Aspekte der Arbeit eingehen. Denn diese ethischen Problematiken, die mit der Nutzung von KI einhergehen, werden nicht sehr ausührlich erläeutert. 

\section{Verbesserungsvorschläge}

Nun würde ich einige Vorschläge zur verbesserung der Arbeit nennen, die beim nächsten Mal helfen können. Du könntest etwas vertiefter darauf eingehen, auf den Datenschutz und die Datenschutzprobleme. Hinzu kommt, dass du eine Art Diskussion oder eine Auflistung machen könntest welchge Risiken es gibt und wie diese den Nutzer der KI und die Gesellschaft beeinflussen könnten. Wie bereits in meiner ''Kritik'' erwähnt, wäre es von Vorteil, wenn du Beispiele von Fallstudien einfügst des Datenschutzes miteinbeziehst, um deinen Inhalt zu stärken oder konkrete Fälle  nennst, indenen Datenschutzprobleme mit KI aufgetreten sind und allenfalls wie diese gelöst wurden. Wenn du die ethische Diskussion erweiterst und mehr auf die ethischen Aspekte einghst (der KI-Nutzung).

\section{Fazit}
Die Arbeit bietet einen guten Überblick über Künstliche Intelligenz (KI) und wie sie trainiert wird. Besonders gut finde ich, dass du die Schritte des KI-Trainings klar erklärt hast. Deine Fragestellung zum Datenschutz ist sehr relevant und wichtig. Allerdings könnte die Arbeit noch tiefer auf die Datenschutzprobleme eingehen.Insgesamt hast du das Thema gut dargestellt und wichtige Punkte angesprochen, aber es gibt noch Möglichkeiten, die Arbeit zu verbessern. Alles in einem finde ich die Arbeit inhaltlich gut und auch der Umfang hat mir gefallen.




\printbibliography

\end{document}

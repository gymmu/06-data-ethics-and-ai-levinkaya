\documentclass{article}

\usepackage[ngerman]{babel}
\usepackage[utf8]{inputenc}
\usepackage[T1]{fontenc}
\usepackage{hyperref}
\usepackage{csquotes}

\usepackage[
    backend=biber,
    style=apa,
    sortlocale=de_DE,
    natbib=true,
    url=false,
    doi=false,
    sortcites=true,
    sorting=nyt,
    isbn=false,
    hyperref=true,
    backref=false,
    giveninits=false,
    eprint=false]{biblatex}
\addbibresource{../references/bibliography.bib}

\title{Review des Papers "Ethik im Umgang mit Daten" von Stefani Anastasova \dots}
\author{Levin Kaya}
\date{\today}

\begin{document}
\maketitle

\abstract{
    Dies ist ein Review der Arbeit zum Thema Ethik im Umgang mit Daten von Stefani Anastasova.


}



\section{Was mir an der Arbeit gut gefällt}
Liebe Stefani, wie ich finde hast du eine umfassende Einführung in die Thematik der KI verfasst und es war auch übersichtlich. Es erklärt gut die grundlegenden Konzepte wie beispielsweise überwachtesnund unüberwachtes Lernen. Ich finde dass gut daher, da es zum Verständnis der Lesers beiträgt.

Auch die Trainingsschritte hast du klar strukturiert. Du hast sehr detailliert beschrieben, wie Training, Validierung und Testen funktioniert. Daher sieht man, dass du diese Prozesse, die eine KI durchlaufen muss, bevor sie einsatzbereit ist verstanden hast.

Ich finde deine Fragestellung sehr schlau gewählt, da die Frage wie KI-Systheme den Datenschutz respektieren können sehr aktuell und von grosser Bedeutung ist.

Du hast ChatGBT als konkretes Beispiel angewendet, somit machst du das Thema greifbarer, denn mehr Leute können sich etwas darunter vorstellen. Somit kann der Leser oder die Lerserin, die theoretischen Konzepte, die du davor erläutert hast, auf ein reales zu übertragen.


Zudem hast du meiner Meinung nach gute Massnahmen zur Sicherrstellung des Datenschutzes gegeben ( Einwilligung, Transparenz, Zweckbindung und Datensicherheit). Sie zeigen praktische Wege zur Lösung der beschriebenen Problemen.

\printbibliography

\end{document}
